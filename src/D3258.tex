% !TeX program = luatex
% !TEX encoding = UTF-8


\documentclass{wg21}

\usepackage{luatexja-fontspec}
%\setmainfont{Noto Sans}
%\setmainjfont{Noto Sans CJK KR}

\title{Formatting of \tcode{charN\_t}}
\docnumber{P3258R0}
\audience{LEWG, SG16}
\author{Corentin Jabot}{corentin.jabot@gmail.com}
\usepackage[dvipsnames]{xcolor}


\begin{document}
\maketitle

\section{Abstract}

We propose to make \tcode{std::format("\{\}", u8"\emoji{🌸}")} well-formed.

\section{Motivation}

Despite having language support for Unicode character types since C++11 and C++20 (for \tcode{char8_t}), the standard library
lacks support for these types, making adoption challenging.
\tcode{std::format} is a great place to start.

Having the ability to format and print sequences of Unicode characters would offer more options when considering the type of
strings produced by reflection.

\begin{colorblock}
int 鳥居;
std::format("{}", meta::name_of(^鳥居)).
\end{colorblock}

But there are many other use cases, ie anyone who tries to use \tcode{charN_t} will need this facility.
It is important for usability, and teachability that the standard library and core language complement each other.
Integration of Unicode character types has been unsatisfactory so far.

\section{Design}

\begin{itemize}
\item \tcode{char8_t}, \tcode{char16_t}, \tcode{char32_t} characters and strings (NTBS, \tcode{std::string}, \tcode{std::string_view}, arrays, ranges) can be
used as a formatted argument to \tcode{std::format}.
\item They cannot be used as the type of the \grammarterm{format string}.
\item We assume the encoding of the output is statically determined by the literal encoding of the \grammarterm{format-string}'s type.
This is how \tcode{format} works already.
\item From there, all string arguments are transcoded:
\begin{itemize}
\item Transcoding a \tcode{char} string is a copy.
\item If the ordinary literal encoding is UTF-8, transcoding a \tcode{char8_t} string is a copy.
\item Otherwise, transcoding is an implementation-defined process that will inject replacement characters for invalid or unrepresentable characters.
\item Lossy conversions is the best we can achieve here.
\end{itemize}
\item Mixing \tcode{char}, \tcode{wchar_t} is still not supported to avoid asking the question of the encoding of the argument.
\item \tcode{std::format(L"\{\}", u8"")} is supported, as this is something Windows users are likely to find useful.
\item Escaping Unicode strings (\tcode{format("\{:?\}", u8"こんにちは")}) works the same as escaping other strings, except characters that have no
representation in the associated literal encoding of the format-string's character type are replaced by \tcode{\textbackslash{}\{xxxx\}}.
\end{itemize}

\subsection{Impact on \tcode{std::print}}

There are 2 scenarios here:
\begin{itemize}
\item For a format string in UTF-N, \tcode{std::format} will convert a \tcode{charN_t} argument to UTF-N, which is lossless, and on a Unicode terminal, the Unicode API will preserve information
(but extra work may be performed)
\item For a format string not in \tcode{UTF-N}, if \tcode{vprint_unicode} is called \textbf{manually}, \tcode{charN_t} will be first converted to the literal encoding, and the whole formatted string might
be converting back to Unicode, potentially losing information.
\end{itemize}

This is fine because calling \tcode{vprint_nonunicode} manually is a limited use case and trying to preserve information here would be extremely challenging:
An implementation would have to transcode the formatting string as it is produced, which would be a very different behavior from \tcode{format}.

Ultimately \tcode{vprint_nonunicode} exists to support Windows terminals and should not be seen as a transcoding facility.
Supporting \tcode{charN_t} \grammarterm{format-string} would be a better way to ensure information preservation when the literal encoding is not UTF-8
(see "future work").

\subsection{\tcode{basic_format_arg} is not modified}

An earlier version of this paper modified \tcode{basic_format_arg} so that the newly added \tcode{charN_t} would be stored in a handle.
However, after talking to Mark de Wever and Jonathan Wakely, we ultimately concluded that:

\begin{itemize}
\item We definitively cannot add all 9 types (\tcode{charN_T}, \tcode{basic_string_view<charN_t>}, \tcode{charN_t*}) as this would be an ABI break in several implementations
(implementations use a custom variant-like type with a discriminant stored in a few bits).
\item We do not care about the cost of type erasing these formatters (one of the motivations to store them in \tcode{basic_format_arg}'s variant), as in general the cost of transcoding would dominate, and we expect them to be less use than, eg the \tcode{char} formatters.
\item Modifying \tcode{basic_format_arg}'s variant would technically be a source break if a user provides an exhaustive (ie non-generic) visitor to \tcode{basic_format_arg::visit}. This is a lesser concern are few users are likely to be impacted.
\item Modifying \tcode{basic_format_arg}'s variant would allow users to provide custom u8 strings values to for example a \tcode{fill} option. It is also unclear how useful this would be.
\end{itemize}

This leaves us with 2 options:
\begin{itemize}
\item Leave \tcode{basic_format_arg} alone (this paper).
\item Add custom handling of \tcode{char8_t*} and \tcode{u8string_view} to \tcode{basic_format_arg}, accepting it would be a potential source break.
This would not be an ABI break for implementation as they have some margin to extend the variant (but it would eat on future extensions).
It is plausible that users would want to parametrize their own formatters with u8 strings arguments. However, there is certainly no motivation to support
\tcode{char16_t}, \tcode{char32_t} there.
\end{itemize}

\section{Implementability}

There is sadly no great way to implement the transcoding part of this proposal with existing standard facilities.
Here are a few implementation strategies and considerations:

\begin{itemize}
\item For an implementation that always uses UTF-8 as the type of character literals (libc++, clang), the only requirement is to weave support for the types,
and add UTF-X->UTF-Y conversion routines, that are not hard to implement and already exist in implementations.
\item For an implementation that supports additional character encodings, there needs to additionally exist a method that converts a UTF-8 sequence to the
literal encoding.
This can be achieved by using iconv or ICU, both of which can create a converter from a name (that one can get from \tcode{std::text_encoding}).
On platforms where this would not be an option, or where no converter for the literal encoding would exist, assuming the execution encoding is a superset
of the literal encoding is a reasonable assumption. For example, \tcode{c32rtomb} can be used.
\end{itemize}

The wording should leave enough room to allow an implementation that always produces a bunch of replacement characters.

Adding \tcode{charN_t} in the various interfaces of format did not present implementation challenges.
Most implementations already support Unicode for escaping and width computation. So the hard work is already done.

\section{Future work}

This paper is minimalist by design, here are a few considerations for the future

\subsection{\tcode{constexpr format}}

It is reasonable to expect \tcode{format} to become \tcode{constexpr} soon-ish.
To make \tcode{charN_t} formatters constexpr, implementation will need some built-in to perform the conversion.
This is a reasonable ask: implementations already need to have conversion routines to evaluate string literal, exposing them to the library
is doable.

We could imagine that built-in to be very similar to an iconv-style interface.
\begin{colorblock}
constexpr size_t
__builtin_utf8_to_ordinary(const char8_t*&, size_t & N, char*& Outbuf, size_t & OutputSize);
\end{colorblock}

(And would use iconv or ICU or whatever transcoding facility the compiler already uses to encode string literals).

\subsection{\tcode{charN_t} format strings}

Ideally, we would support \tcode{charN_t} format strings (\tcode{std::format(u8"\{\}", "")}).
However, the following questions would need to be answered:
\begin{itemize}
\item How does \tcode{to_chars} and \tcode{char8_t} interract? Unicode has a large set of numbers.
\item Are existing locale facilities sufficient to support the needs of Unicode?
\item What do we assume the encoding for \tcode{char} and \tcode{wchar_t} to be?
\item What is the implementation burden?
\item What is the interaction with user-defined formatters?
\end{itemize}

\subsection{\tcode{wchar_t} <-> \tcode{char}}

We do not propose to allow implicit transcoding between \tcode{wchar_t} and \tcode{char}.
This is because we would need a better understanding of the nature of formatting arguments (execution or literal encoding?),
and there seems to be less demand for it.

\subsection{Error handling options}

We could imagine letting the user control the replacement character to use or whether to throw a formatting error on non-representable/invalid characters.
This can be explored separately as it would impact existing character types.

\subsection{Better specification}

As both C and C++ gain better transcoding facilities in future standards, we can respecify in terms of these facilities.

\subsection{What about \tcode{iostream}?}

This is a story for another paper (One that an enthusiastic reader is encouraged to write!)

\section{Wording}

\rSec1[basic.types]{Types}%
\rSec2[basic.fundamental]{Fundamental types}


\pnum
\indextext{Boolean type}%
\indextext{type!Boolean}%
Type \tcode{bool} is a distinct type that has
the same object representation,
value representation, and
alignment requirements as
an \impldef{underlying type of \tcode{bool}} unsigned integer type.
The values of type \keyword{bool} are
\keyword{true} and \keyword{false}.
\begin{note}
    There are no \keyword{signed}, \keyword{unsigned},
    \keyword{short}, or \tcode{\keyword{long} \keyword{bool}} types or values.
\end{note}

\pnum
\indextext{type!integral}%
\added{The types \keyword{char8_t}, \keyword{char16_t}, and \keyword{char32_t}
    are collectively called \defnadjx{Unicode character}{types}{type}.}
The \added{The Unicode character types}, \keyword{char},  \added{and} \keyword{wchar_t}\removed{,
\keyword{char8_t}, \keyword{char16_t}, and \keyword{char32_t}}
are collectively called \defnadjx{character}{types}{type}.
The character types, \keyword{bool},
the signed and unsigned integer types,
and cv-qualified versions\iref{basic.type.qualifier} thereof,
are collectively termed
\defnx{integral types}{integral type}.
A synonym for integral type is \defn{integer type}.
\begin{note}
    Enumerations\iref{dcl.enum} are not integral;
    however, unscoped enumerations can be promoted to integral types
    as specified in \ref{conv.prom}.
\end{note}

\rSec1[format]{Formatting}

\rSec3[format.string.std]{Standard format specifiers}

\pnum
The \fmtgrammarterm{align} option applies to all argument types.
The meaning of the various alignment options is as specified in \tref{format.align}.
\begin{note}
    The \fmtgrammarterm{fill}, \fmtgrammarterm{align}, and \tcode{0} options
    have no effect when the minimum field width
    is not greater than the estimated field width
    because padding width is \tcode{0} in that case.
    Since fill characters are assumed to have a field width of \tcode{1},
    use of a character with a different field width can produce misaligned output.
    The UNICODE CLOWN FACE character has a field width of \tcode{2}.
    The examples above that include that character
    illustrate the effect of the field width
    when that character is used as a fill character
    as opposed to when it is used as a formatting argument.
\end{note}

\begin{floattable}{Meaning of \fmtgrammarterm{align} options}{format.align}{lp{.8\hsize}}
    \topline
    \lhdr{Option} & \rhdr{Meaning} \\ \rowsep
    \tcode{<} &
    Forces the formatted argument to be aligned to the start of the field
    by inserting $n$ fill characters after the formatted argument
    where $n$ is the padding width.
    This is the default for
    non-arithmetic non-pointer types, \added{the Unicode character types}, \tcode{charT}, and \tcode{bool},
    unless an integer presentation type is specified.
    \\ \rowsep
    %
    \tcode{>} &
    Forces the formatted argument to be aligned to the end of the field
    by inserting $n$ fill characters before the formatted argument
    where $n$ is the padding width.
    This is the default for
    arithmetic types other than \added{the Unicode character types}, \tcode{charT} and \tcode{bool},
    pointer types,
    or when an integer presentation type is specified.
    \\ \rowsep
    %
    \tcode{\caret} &
    Forces the formatted argument to be centered within the field
    by inserting
    $\bigl\lfloor \frac{n}{2} \bigr\rfloor$
    fill characters before and
    $\bigl\lceil \frac{n}{2} \bigr\rceil$
    fill characters after the formatted argument, where
    $n$ is the padding width.
    \\
\end{floattable}

\pnum
The \fmtgrammarterm{sign} option is only valid
for arithmetic types other than \added{the Unicode character types}, \tcode{charT} and \tcode{bool}
or when an integer presentation type is specified.
The meaning of the various options is as specified in \tref{format.sign}.


\pnum
% FIXME: What is a "string" here, exactly?
The available string presentation types are specified in \tref{format.type.string}.
%
\begin{floattable}{Meaning of \fmtgrammarterm{type} options for strings}{format.type.string}{ll}
    \topline
    \lhdr{Type} & \rhdr{Meaning} \\ \rowsep
    none, \tcode{s} &
    Copies the \added{transcoded \iref{format.string.transcoded}} string to the output.
    \\ \rowsep
    %
    \tcode{?} &
    Copies the escaped string \iref{format.string.escaped} to the output.
    \\
\end{floattable}


\begin{floattable}{Meaning of \fmtgrammarterm{type} options for integer types}{format.type.int}{lp{.8\hsize}}
\topline
\lhdr{Type} & \rhdr{Meaning} \\ \rowsep
\tcode{b} &
\tcode{to_chars(first, last, value, 2)};
\indextext{base prefix}%
the base prefix is \tcode{0b}.
\\ \rowsep
%
\tcode{B} &
The same as \tcode{b}, except that
\indextext{base prefix}%
the base prefix is \tcode{0B}.
\\ \rowsep
%
\tcode{c} &
Copies the character \tcode{static_cast<charT>(value)} to the output.
Throws \tcode{format_error} if \tcode{value} is not
in the range of representable values for \tcode{charT}.
\\ \rowsep
%
\tcode{d} &
\tcode{to_chars(first, last, value)}.
\\ \rowsep
%
\tcode{o} &
\tcode{to_chars(first, last, value, 8)};
\indextext{base prefix}%
the base prefix is \tcode{0} if \tcode{value} is nonzero and is empty otherwise.
\\ \rowsep
%
\tcode{x} &
\tcode{to_chars(first, last, value, 16)};
\indextext{base prefix}%
the base prefix is \tcode{0x}.
\\ \rowsep
%
\tcode{X} &
The same as \tcode{x}, except that
it uses uppercase letters for digits above 9 and
\indextext{base prefix}%
the base prefix is \tcode{0X}.
\\ \rowsep
%
none &
The same as \tcode{d}.
\begin{note}
    If the formatting argument type is \tcode{charT}\added{, a Unicode character type,} or \tcode{bool},
    the default is instead \tcode{c} or \tcode{s}, respectively.
\end{note}
\\
\end{floattable}

\begin{floattable}{Meaning of \fmtgrammarterm{sign} options}{format.sign}{lp{.8\hsize}}
    \topline
    \lhdr{Option} & \rhdr{Meaning} \\ \rowsep
    \tcode{+} &
    Indicates that a sign should be used for both non-negative and negative
    numbers.
    The \tcode{+} sign is inserted before the output of \tcode{to_chars} for
    non-negative numbers other than negative zero.
    \begin{note}
        For negative numbers and negative zero
        the output of \tcode{to_chars} will already contain the sign
        so no additional transformation is performed.
    \end{note}
    \\ \rowsep
    %
    \tcode{-} &
    Indicates that a sign should be used for
    negative numbers and negative zero only (this is the default behavior).
    \\ \rowsep
    %
    space &
    Indicates that a leading space should be used for
    non-negative numbers other than negative zero, and
    a minus sign for negative numbers and negative zero.
    \\
\end{floattable}

\pnum
The \fmtgrammarterm{sign} option applies to floating-point infinity and NaN.
\begin{example}
    \begin{codeblock}
        double inf = numeric_limits<double>::infinity();
        double nan = numeric_limits<double>::quiet_NaN();
        string s0 = format("{0:},{0:+},{0:-},{0: }", 1);        // value of \tcode{s0} is \tcode{"1,+1,1, 1"}
        string s1 = format("{0:},{0:+},{0:-},{0: }", -1);       // value of \tcode{s1} is \tcode{"-1,-1,-1,-1"}
        string s2 = format("{0:},{0:+},{0:-},{0: }", inf);      // value of \tcode{s2} is \tcode{"inf,+inf,inf, inf"}
        string s3 = format("{0:},{0:+},{0:-},{0: }", nan);      // value of \tcode{s3} is \tcode{"nan,+nan,nan, nan"}
    \end{codeblock}
\end{example}

\pnum
The \tcode{\#} option causes the
% FIXME: This is not a definition.
\defnx{alternate form}{alternate form!format string}
to be used for the conversion.
This option is valid for arithmetic types other than
\added{Unicode character types,} \tcode{charT} and \tcode{bool}
or when an integer presentation type is specified, and not otherwise.
For integral types,
the alternate form inserts the
base prefix (if any) specified in \tref{format.type.int}
into the output after the sign character (possibly space) if there is one, or
before the output of \tcode{to_chars} otherwise.
For floating-point types,
the alternate form causes the result of the conversion of finite values
to always contain a decimal-point character,
even if no digits follow it.
% FIXME: This is a weird place for this part of the spec to appear.
Normally, a decimal-point character appears in the result of these
conversions only if a digit follows it.
In addition, for \tcode{g} and \tcode{G} conversions,
% FIXME: Are they normally? What does this even mean? Reach into to_chars and
% alter its behavior?
trailing zeros are not removed from the result.

\pnum
The \tcode{0} option is valid for arithmetic types
other than \added{Unicode character types,} \tcode{charT} and \tcode{bool}, pointer types, or
when an integer presentation type is specified.
For formatting arguments that have a value
other than an infinity or a NaN,
this option pads the formatted argument by
inserting the \tcode{0} character $n$ times
following the sign or base prefix indicators (if any)
where $n$ is \tcode{0} if the \fmtgrammarterm{align} option is present and
is the padding width otherwise.
\begin{example}
    \begin{codeblock}
        char c = 120;
        string s1 = format("{:+06d}", c);       // value of \tcode{s1} is \tcode{"+00120"}
        string s2 = format("{:#06x}", 0xa);     // value of \tcode{s2} is \tcode{"0x000a"}
        string s3 = format("{:<06}", -42);      // value of \tcode{s3} is \tcode{"-42\ \ \ "} (\tcode{0} has no effect)
        string s4 = format("{:06}", inf);       // value of \tcode{s4} is \tcode{"\ \ \ inf"} (\tcode{0} has no effect)
    \end{codeblock}
\end{example}

\pnum
The \fmtgrammarterm{width} option specifies the minimum field width.
If the \fmtgrammarterm{width} option is absent,
the minimum field width is \tcode{0}.

\pnum
If \tcode{\{ \opt{\fmtgrammarterm{arg-id}} \}} is used in
a \fmtgrammarterm{width} or \fmtgrammarterm{precision} option,
the value of the corresponding formatting argument is used as the value of the option.
The option is valid only if the corresponding formatting argument is
of standard signed or unsigned integer type.
If its value is negative,
an exception of type \tcode{format_error} is thrown.

\pnum
% FIXME: What if it's an arg-id?
If \fmtgrammarterm{positive-integer} is used in a
\fmtgrammarterm{width} option, the value of the \fmtgrammarterm{positive-integer}
is interpreted as a decimal integer and used as the value of the option.

\pnum
For the purposes of width computation, \changed{a}{an ordinary or wide character} string is assumed to be in
a locale-independent, \impldef{encoding assumption for \tcode{format} width computation} encoding.
Implementations should use either UTF-8, UTF-16, or UTF-32, on platforms capable of displaying Unicode text in a terminal.
\begin{note}
    This is the case for Windows-based and many POSIX-based operating systems.
\end{note}

\begin{addedblock}
\begin{note}
    \tcode{char8_t}, \tcode{char16_t}, \tcode{char32_t} strings are assumed to be in UTF-8, UTF-16, or UTF-32 respectively.
\end{note}
\end{addedblock}

\pnum
The available \added{Unicode character type} anf \tcode{charT} presentation types are specified in \ednote{the table below}
%
\begin{floattable}{Meaning of \fmtgrammarterm{type} options for \tcode{charT}}{format.type.char}{lp{.8\hsize}}
    \topline
    \lhdr{Type} & \rhdr{Meaning} \\ \rowsep
    none, \tcode{c} &
    Copies the \added{transcoded \iref{format.string.transcoded} }character to the output.
    \\ \rowsep
    %
    \tcode{b}, \tcode{B}, \tcode{d}, \tcode{o}, \tcode{x}, \tcode{X} &
    As specified in \tref{format.type.int}
    with \tcode{value} converted to the unsigned version of the underlying type.
    \\ \rowsep
    %
    \tcode{?} &
    Copies the escaped character \iref{format.string.escaped} to the output.
    \\
\end{floattable}


\rSec3[format.formatter.spec]{Formatter specializations}
\indexlibraryglobal{formatter}%

\pnum
% FIXME: Specify this in [format.functions], not here!
The functions defined in \ref{format.functions} use
specializations of the class template \tcode{formatter} to format
individual arguments.

\pnum
Let \tcode{charT} be either \tcode{char} or \keyword{wchar_t}.
Each specialization of \tcode{formatter} is either enabled or disabled,
as described below.
\indextext{\idxcode{formatter}!debug-enabled specialization of}%
A \defn{debug-enabled} specialization of \tcode{formatter}
additionally provides
a public, constexpr, non-static member function \tcode{set_debug_format()}
which modifies the state of the \tcode{formatter} to be as if
the type of the \fmtgrammarterm{std-format-spec}
parsed by the last call to \tcode{parse} were \tcode{?}.
Each header that declares the template \tcode{formatter}
provides the following enabled specializations:
\begin{itemize}
    \item
    \indexlibrary{\idxcode{formatter}!specializations!character types}%
    The debug-enabled specializations
    \begin{codeblock}
        template<> struct formatter<char, char>;
        template<> struct formatter<char, wchar_t>;
        template<> struct formatter<wchar_t, wchar_t>;\end{codeblock}\begin{addedblock}\begin{codeblock}
        template<> struct formatter<char8_t, char>;
        template<> struct formatter<char16_t, char>;
        template<> struct formatter<char32_t, char>;
        template<> struct formatter<char8_t,  wchar_t>;
        template<> struct formatter<char16_t, wchar_t>;
        template<> struct formatter<char32_t, wchar_t>;
    \end{codeblock}
    \end{addedblock}

    \item
    \indexlibrary{\idxcode{formatter}!specializations!string types}%
    For each \tcode{charT}, \added{for each Unicode character type UcharT},
    the debug-enabled string type specializations
    \begin{codeblock}
        template<> struct formatter<charT*, charT>;
        template<> struct formatter<const charT*, charT>;
        template<size_t N> struct formatter<charT[N], charT>;
        template<class traits, class Allocator>
        struct formatter<basic_string<charT, traits, Allocator>, charT>;
        template<class traits>
        struct formatter<basic_string_view<charT, traits>, charT>;\end{codeblock}\begin{addedblock}\begin{codeblock}
        template<> struct formatter<UcharT*, charT>;
        template<> struct formatter<const UcharT*, charT>;
        template<size_t N> struct formatter<UcharT[N], charT>;
        template<class traits, class Allocator>
        struct formatter<basic_string<UcharT, traits, Allocator>, charT>;
        template<class traits>
        struct formatter<basic_string_view<UcharT, traits>, charT>;
        \end{codeblock}
    \end{addedblock}

    \item
    \indexlibrary{\idxcode{formatter}!specializations!arithmetic types}%
    For each \tcode{charT},
    for each cv-unqualified arithmetic type \tcode{ArithmeticT}
    other than
    \tcode{char},
    \keyword{wchar_t},
    \keyword{char8_t},
    \keyword{char16_t}, or
    \keyword{char32_t},
    a specialization
    \begin{codeblock}
        template<> struct formatter<ArithmeticT, charT>;
    \end{codeblock}

    \item
    \indexlibrary{\idxcode{formatter}!specializations!pointer types}%
    \indexlibrary{\idxcode{formatter}!specializations!\idxcode{nullptr_t}}%
    For each \tcode{charT},
    the pointer type specializations
    \begin{codeblock}
        template<> struct formatter<nullptr_t, charT>;
        template<> struct formatter<void*, charT>;
        template<> struct formatter<const void*, charT>;
    \end{codeblock}
\end{itemize}
The \tcode{parse} member functions of these formatters
interpret the format specification
as a \fmtgrammarterm{std-format-spec}
as described in \ref{format.string.std}.
In addition,
for each type \tcode{T} for which
a \tcode{formatter} specialization is provided above,
each of the headers provides the following specialization:
\begin{codeblock}
    template<> inline constexpr bool enable_nonlocking_formatter_optimization<T> = true;
\end{codeblock}
\begin{note}
    Specializations such as \tcode{formatter<wchar_t, char>}
    and \tcode{formatter<const char*, wchar_t>}
    that would require implicit
    multibyte / wide string or character conversion are disabled.
\end{note}

\pnum
For any types \tcode{T} and \tcode{charT} for which
neither the library nor the user provides
an explicit or partial specialization of
the class template \tcode{formatter},
\tcode{formatter<T, charT>} is disabled.

\pnum
If the library provides an explicit or partial specialization of
\tcode{formatter<T, charT>}, that specialization is enabled
and meets the \oldconcept{Formatter} requirements
except as noted otherwise.

\pnum
If \tcode{F} is a disabled specialization of \tcode{formatter}, these
values are \tcode{false}:
\begin{itemize}
    \item \tcode{is_default_constructible_v<F>},
    \item \tcode{is_copy_constructible_v<F>},
    \item \tcode{is_move_constructible_v<F>},
    \item \tcode{is_copy_assignable_v<F>}, and
    \item \tcode{is_move_assignable_v<F>}.
\end{itemize}

\pnum
An enabled specialization \tcode{formatter<T, charT>} meets the
\oldconcept{BasicFormatter} requirements \iref{formatter.requirements}.
\begin{example}
    \begin{codeblock}
        #include <format>
        #include <string>

        enum color { red, green, blue };
        const char* color_names[] = { "red", "green", "blue" };

        template<> struct std::formatter<color> : std::formatter<const char*> {
            auto format(color c, format_context& ctx) const {
                return formatter<const char*>::format(color_names[c], ctx);
            }
        };

        struct err {};

        std::string s0 = std::format("{}", 42);         // OK, library-provided formatter
        std::string s1 = std::format("{}", L"foo");     // error: disabled formatter
        std::string s2 = std::format("{}", red);        // OK, user-provided formatter
        std::string s3 = std::format("{}", err{});      // error: disabled formatter
    \end{codeblock}
\end{example}

\begin{addedblock}
\rSec3[format.string.transcoded]{Formatting transcoded characters and strings}

The transcoded string \placeholder{E} representation of a string \placeholder{S}
is constructed by encoding a sequence of characters as follows:


Let \placeholder{ReplacementCharacter} be an implementation-defined code unit sequence in the the associated character encoding \placeholder{TE} for \tcode{charT}  ([lex.string.literal])

If \placeholder{TE} is the same as the associated encoding \placeholder{SE} of the character type \placeholder{ST} of \placeholder{S} \ednote{This needs to be defined},
each code unit of  \placeholder{S} is appended to E.

Otherwise (\placeholder{ST} is a Unicode character type), for each code unit sequence \placeholder{X} in \placeholder{S} that either
encodes a single character, or is a sequence of ill-formed code units,
processing is in order as follows:
\begin{itemize}
\item If \placeholder{X} encodes a single character \placeholder{C}, then, if there exist an implementation-defined representation \placeholder{R}
of \placeholder{C} in \placeholder{TE}, each code unit of \placeholder{R} is appended to \placeholder{E}. If no such representation exists, \placeholder{ReplacementCharacter} is appended to \placeholder{E}.
\item Otherwise \placeholder{ReplacementCharacter} is appended to \placeholder{E}.
\end{itemize}

The transcoded string representation of a character \placeholder{C}
is equivalent to the transcoded string representation
of a string of \placeholder{C}.

\end{addedblock}

\rSec3[format.string.escaped]{Formatting escaped characters and strings}

\pnum
\indextext{string!formatted as escaped}%
\indextext{character!formatted as escaped}%
A character or string can be formatted as \defn{escaped}
to make it more suitable for debugging or for logging.

\pnum
The escaped string \placeholder{E} representation of a string \placeholder{S}
is constructed by encoding a sequence of characters as follows.

\removed{The associated character encoding \placeholder{CE} for \tcode{charT}~(\tref{lex.string.literal})
is used to both interpret \placeholder{S} and construct \placeholder{E}.}

\added{Let \placeholder{TE} be the associated character encoding for \tcode{charT}.}
\added{Let \placeholder{SE} be the associated character encoding for the character type of \placeholder{S}.}

\begin{itemize}
\item
\unicode{0022}{quotation mark} (\tcode{"}) is appended to \placeholder{E}.

\item
For each code unit sequence \placeholder{X} in \placeholder{S} that either
encodes a single character,
is a shift sequence, or
is a sequence of ill-formed code units,
processing is in order as follows:

\begin{itemize}
\item
If \placeholder{X} encodes a single character \placeholder{C}, then:

\begin{itemize}
    \item
    If \placeholder{C} is one of the characters in \tref{format.escape.sequences},
    then the two characters shown as the corresponding escape sequence
    are appended to \placeholder{E}.

    \item
    Otherwise, if \placeholder{C} is not \unicode{0020}{space} and

    \begin{itemize}
        \item
        \placeholder{\changed{CE}{SE}} is UTF-8, UTF-16, or UTF-32 and
        \placeholder{C} corresponds to a Unicode scalar value
        whose Unicode property \tcode{General_Category} has a value in the groups
        \tcode{Separator} (\tcode{Z}) or \tcode{Other} (\tcode{C}),
        as described by UAX 44 of the Unicode Standard, or

        \item
        \placeholder{\changed{CE}{SE}} is UTF-8, UTF-16, or UTF-32 and
        \placeholder{C} corresponds to a Unicode scalar value
        with the Unicode property \tcode{Grapheme_Extend=Yes}
        as described by UAX 44 of the Unicode Standard and
        \placeholder{C} is not immediately preceded in \placeholder{S} by
        a character \placeholder{P} appended to \placeholder{E}
        without translation to an escape sequence, or

        \item
        \placeholder{\changed{CE}{SE}} is neither UTF-8, UTF-16, nor UTF-32 and
        \placeholder{C} is one of an implementation-defined set
        of separator or non-printable characters

        \begin{addedblock}
        \item
        \placeholder{SE} and \placeholder{TE} do not denote the same encoding and
        \placeholder{C} has no implementation-defined representation in \placeholder{TE}
        \end{addedblock}

    \end{itemize}

    then the sequence \tcode{\textbackslash u\{\placeholder{hex-digit-sequence}\}}
    is appended to \placeholder{E},
    where \tcode{\placeholder{hex-digit-sequence}}
    is the shortest hexadecimal representation
    of \placeholder{C} using lower-case hexadecimal digits.

    \begin{addedblock}
    Otherwise, if \placeholder{SE} and \placeholder{TE} do not denote the same encoding,
    the implementation-defined representation of \placeholder{C} in \placeholder{TE} is appended to \placeholder{E}.
    \end{addedblock}

    \item
    Otherwise, \placeholder{C} is appended to \placeholder{E}.
\end{itemize}

\item
Otherwise, if \placeholder{X} is a shift sequence,
the effect on \placeholder{E} and further decoding of \placeholder{S}
is unspecified.

\recommended
A shift sequence should be represented in \placeholder{E}
such that the original code unit sequence of \placeholder{S}
can be reconstructed.

\item
Otherwise (\placeholder{X} is a sequence of ill-formed code units),
each code unit \placeholder{U} is appended to \placeholder{E} in order
as the sequence \tcode{\textbackslash x\{\placeholder{hex-digit-sequence}\}},
where \tcode{\placeholder{hex-digit-sequence}}
is the shortest hexadecimal representation of \placeholder{U}
using lower-case hexadecimal digits.
\end{itemize}

\item
Finally, \unicode{0022}{quotation mark} (\tcode{"})
is appended to \placeholder{E}.
\end{itemize}
%
\begin{floattable}{Mapping of characters to escape sequences}{format.escape.sequences}{ll}
    \topline
    \lhdr{Character} & \rhdr{Escape sequence} \\ \rowsep
    \unicode{0009}{character tabulation} &
    \tcode{\textbackslash t}
    \\ \rowsep
    %
    \unicode{000a}{line feed} &
    \tcode{\textbackslash n}
    \\ \rowsep
    %
    \unicode{000d}{carriage return} &
    \tcode{\textbackslash r}
    \\ \rowsep
    %
    \unicode{0022}{quotation mark} &
    \tcode{\textbackslash "}
    \\ \rowsep
    %
    \unicode{005c}{reverse solidus} &
    \tcode{\textbackslash\textbackslash}
    \\
\end{floattable}

\pnum
The escaped string representation of a character \placeholder{C}
is equivalent to the escaped string representation
of a string of \placeholder{C}, except that:

\begin{itemize}
    \item
    the result starts and ends with \unicode{0027}{apostrophe} (\tcode{'})
    instead of \unicode{0022}{quotation mark} (\tcode{"}), and
    \item
    if \placeholder{C} is \unicode{0027}{apostrophe},
    the two characters \tcode{\textbackslash '} are appended to \placeholder{E}, and
    \item
    if \placeholder{C} is \unicode{0022}{quotation mark},
    then \placeholder{C} is appended unchanged.
\end{itemize}

\begin{example}
%    \begin{codeblock}
%        string s0 = format("[{}]", "h\tllo");                   // \tcode{s0} has value: \tcode{[h\ \ \ \ llo]}
%        string s1 = format("[{:?}]", "h\tllo");                 // \tcode{s1} has value: \tcode{["h\textbackslash tllo"]}
%        string s2 = format("[{:?}]", "@\importexample[-2.5pt]{example_01}@");  @\kern1.25pt@// \tcode{s2} has value: \tcode{["\importexample[-2.5pt]{example_01}"]}
%        string s3 = format("[{:?}, {:?}]", '\'', '"');          // \tcode{s3} has value: \tcode{['\textbackslash '', '"']}
%
%        // The following examples assume use of the UTF-8 encoding
%        string s4 = format("[{:?}]", string("\0 \n \t \x02 \x1b", 9));
%        // \tcode{s4} has value: \tcode{["\textbackslash u\{0\} \textbackslash n \textbackslash t \textbackslash u\{2\} \textbackslash u\{1b\}"]}
%        string s5 = format("[{:?}]", "\xc3\x28");           // invalid UTF-8, \tcode{s5} has value: \tcode{["\textbackslash x\{c3\}("]}
%        string s6 = format("[{:?}]", "@\importexample{example_02}@");                 @\kern0.75pt@// \tcode{s6} has value: \tcode{["\importexample{example_03}\textbackslash{u}\{200d\}\importexample{example_04}"]}
%        string s7 = format("[{:?}]", "\u0301");             // \tcode{s7} has value: \tcode{["\textbackslash u\{301\}"]}
%        string s8 = format("[{:?}]", "\\\u0301");           // \tcode{s8} has value: \tcode{["\textbackslash \textbackslash \textbackslash u\{301\}"]}
%        string s9 = format("[{:?}]", "e\u0301\u0323");      // \tcode{s9} has value: \tcode{["\importexample[-2pt]{example_06}"]}
%    \end{codeblock}
\end{example}

\ednote{[...]}


\rSec3[format.range.formatter]{Class template \tcode{range_formatter}}

\pnum
The class template \tcode{range_formatter} is a utility
for implementing \tcode{formatter} specializations for range types.

\pnum
\tcode{range_formatter} interprets \fmtgrammarterm{format-spec}
as a \fmtgrammarterm{range-format-spec}.
The syntax of format specifications is as follows:

\begin{ncbnf}
    \fmtnontermdef{range-format-spec}\br
    \opt{range-fill-and-align} \opt{width} \opt{\terminal{n}} \opt{range-type} \opt{range-underlying-spec}
\end{ncbnf}

\begin{ncbnf}
    \fmtnontermdef{range-fill-and-align}\br
    \opt{range-fill} align
\end{ncbnf}

\begin{ncbnf}
    \fmtnontermdef{range-fill}\br
    \textnormal{any character other than} \terminal{\{} \textnormal{or} \terminal{\}} \textnormal{or} \terminal{:}
\end{ncbnf}

\begin{ncbnf}
    \fmtnontermdef{range-type}\br
    \terminal{m}\br
    \terminal{s}\br
    \terminal{?s}
\end{ncbnf}

\begin{ncbnf}
    \fmtnontermdef{range-underlying-spec}\br
    \terminal{:} format-spec
\end{ncbnf}

\pnum
For \tcode{range_formatter<T, charT>},
the \fmtgrammarterm{format-spec}
in a \fmtgrammarterm{range-underlying-spec}, if any,
is interpreted by \tcode{formatter<T, charT>}.

\pnum
The \fmtgrammarterm{range-fill-and-align} is interpreted
the same way as a \fmtgrammarterm{fill-and-align}\iref{format.string.std}.
The productions \fmtgrammarterm{align} and \fmtgrammarterm{width}
are described in \ref{format.string}.

\pnum
The \tcode{n} option causes the range to be formatted
without the opening and closing brackets.
\begin{note}
    This is equivalent to invoking \tcode{set_brackets(\{\}, \{\})}.
\end{note}

\pnum
The \fmtgrammarterm{range-type} specifier changes the way a range is formatted,
with certain options only valid with certain argument types.
The meaning of the various type options
is as specified in \tref{formatter.range.type}.

\begin{concepttable}{Meaning of \fmtgrammarterm{range-type} options}{formatter.range.type}
    {p{1in}p{1.4in}p{2.7in}}
    \topline
    \hdstyle{Option} & \hdstyle{Requirements} & \hdstyle{Meaning} \\ \capsep
    %
    \tcode{m} &
    \tcode{T} shall be
    either a specialization of \tcode{pair} or a specialization of \tcode{tuple}
    such that \tcode{tuple_size_v<T>} is \tcode{2}. &
    Indicates that
    the opening bracket should be \tcode{"\{"},
    the closing bracket should be \tcode{"\}"},
    the separator should be \tcode{", "}, and
    each range element should be formatted as if
    \tcode{m} were specified for its \fmtgrammarterm{tuple-type}.
    \begin{note}
        If the \tcode{n} option is provided in addition to the \tcode{m} option,
        both the opening and closing brackets are still empty.
    \end{note}
    \\ \rowsep
    %
    \tcode{s} &
    \tcode{T} shall be \added{a Unicode character type or} \tcode{charT}. &
    Indicates that the range should be formatted as \changed{a}{a transcoded string \iref{format.string.transcoded}} \tcode{string}.
    \\ \rowsep
    %
    \tcode{?s} &
    \tcode{T} shall be \added{a Unicode character type or} \tcode{charT}. &
    Indicates that the range should be formatted as
    an escaped string \iref{format.string.escaped}.
    \\
\end{concepttable}

If the \fmtgrammarterm{range-type} is \tcode{s} or \tcode{?s},
then there shall be
no \tcode{n} option and no \fmtgrammarterm{range-underlying-spec}.

\indexlibrarymember{set_separator}{range_formatter}%
\begin{itemdecl}
constexpr void set_separator(basic_string_view<charT> sep) noexcept;
\end{itemdecl}

\begin{itemdescr}
\pnum
\effects
Equivalent to: \tcode{\exposid{separator_} = sep;}
\end{itemdescr}

\indexlibrarymember{set_brackets}{range_formatter}%
\begin{itemdecl}
constexpr void set_brackets(basic_string_view<charT> opening,
basic_string_view<charT> closing) noexcept;
\end{itemdecl}

\begin{itemdescr}
\pnum
\effects
Equivalent to:
\begin{codeblock}
    @\exposid{opening-bracket\_}@ = opening;
    @\exposid{closing-bracket\_}@ = closing;
\end{codeblock}
\end{itemdescr}

\indexlibrarymember{parse}{range_formatter}%
\begin{itemdecl}
template<class ParseContext>
constexpr typename ParseContext::iterator
parse(ParseContext& ctx);
\end{itemdecl}

\begin{itemdescr}
\pnum
\effects
Parses the format specifiers as a \fmtgrammarterm{range-format-spec} and
stores the parsed specifiers in \tcode{*this}.
Calls \tcode{\exposid{underlying_}.parse(ctx)} to parse
\fmtgrammarterm{format-spec} in \fmtgrammarterm{range-format-spec} or,
if the latter is not present, an empty \fmtgrammarterm{format-spec}.
The values of
\exposid{opening-bracket_}, \exposid{closing-bracket_}, and \exposid{separator_}
are modified if and only if required by
the \fmtgrammarterm{range-type} or the \tcode{n} option, if present.
If:
\begin{itemize}
    \item
    the \fmtgrammarterm{range-type} is neither \tcode{s} nor \tcode{?s},
    \item
    \tcode{\exposid{underlying_}.set_debug_format()} is a valid expression, and
    \item
    there is no \fmtgrammarterm{range-underlying-spec},
\end{itemize}
then calls \tcode{\exposid{underlying_}.set_debug_format()}.

\pnum\textbf{}
\returns
An iterator past the end of the \fmtgrammarterm{range-format-spec}.
\end{itemdescr}

\indexlibrarymember{format}{range_formatter}%
\begin{itemdecl}
template<ranges::@\libconcept{input\_range}@ R, class FormatContext>
requires @\libconcept{formattable}@<ranges::range_reference_t<R>, charT> &&
@\libconcept{same\_as}@<remove_cvref_t<ranges::range_reference_t<R>>, T>
typename FormatContext::iterator
format(R&& r, FormatContext& ctx) const;
\end{itemdecl}

\begin{itemdescr}
\pnum
\effects
Writes the following into \tcode{ctx.out()},
adjusted according to the \fmtgrammarterm{range-format-spec}:

\begin{itemize}
    \item
    If the \fmtgrammarterm{range-type} was \tcode{s},
    then as if by formatting \tcode{basic_string<charT>(from_range, r)}
    \added {as a transcoded string \iref{format.string.transcoded}}.
    \item
    Otherwise, if the \fmtgrammarterm{range-type} was \tcode{?s},
    then as if by formatting \tcode{basic_string<charT>(from_range, r)}
    as an escaped string \iref{format.string.escaped}.
    \item
    Otherwise,
    \begin{itemize}
        \item
        \exposid{opening-bracket_},
        \item
        for each element \tcode{e} of the range \tcode{r}:
        \begin{itemize}
            \item
            the result of writing \tcode{e} via \exposid{underlying_} and
            \item
            \exposid{separator_}, unless \tcode{e} is the last element of \tcode{r}, and
        \end{itemize}
        \item
        \exposid{closing-bracket_}.
    \end{itemize}
\end{itemize}

\pnum
\returns
An iterator past the end of the output range.
\end{itemdescr}


\rSec3[format.range.fmtstr]{Specialization of \exposid{range-default-formatter} for strings}

\indexlibrary{range-default-formatter@\exposid{range-default-formatter}}%
\begin{codeblock}
namespace std {
template<range_format K, ranges::@\libconcept{input\_range}@ R, class charT>
requires (K == range_format::string || K == range_format::debug_string)
struct @\exposid{range-default-formatter}@<K, R, charT> {
    private:
    @\added{using char\_type = remove\_cvref\_t<range\_reference\_t<R>;// \expos}@
    formatter<basic_string<@\changed{charT}{char\_type}@>, charT> @\exposid{underlying\_}@;

    public:
    template<class ParseContext>
    constexpr typename ParseContext::iterator
    parse(ParseContext& ctx);

    template<class FormatContext>
    typename FormatContext::iterator
    format(@\seebelow@& str, FormatContext& ctx) const;
};
}
\end{codeblock}

\pnum
\mandates
\changed{\tcode{\libconcept{same\_as}<remove_cvref_t<range_reference_t<R>>, charT>} is \tcode{true}}{\\
    \tcode{char_type} denotes \tcode{charT} or a Unicode character type}.



\begin{itemdecl}
    template<class ParseContext>
    constexpr typename ParseContext::iterator
    parse(ParseContext& ctx);
\end{itemdecl}

\begin{itemdescr}
    \pnum
    \effects
    Equivalent to:
    \begin{codeblock}
        auto i = @\exposid{underlying\_}@.parse(ctx);
        if constexpr (K == range_format::debug_string) {
            @\exposid{underlying\_}@.set_debug_format();
        }
        return i;
    \end{codeblock}
\end{itemdescr}

\begin{itemdecl}
    template<class FormatContext>
    typename FormatContext::iterator
    format(@\seebelow@& r, FormatContext& ctx) const;
\end{itemdecl}

\begin{itemdescr}
    \pnum
    The type of \tcode{r} is \tcode{const R\&}
    if \tcode{ranges::\libconcept{input\_range}<const R>} is \tcode{true} and
    \tcode{R\&} otherwise.

    \pnum
    \effects
    Let \tcode{\placeholder{s}} be a \tcode{basic_string<\changed{charT}{char_type}>} such that
    \tcode{ranges::equal(\placeholder{s}, r)} is \tcode{true}.
    Equivalent to: \tcode{return \exposid{underlying_}.format(\placeholder{s}, ctx);}
\end{itemdescr}

%\rSec2[format.arguments]{Arguments}
%
%\rSec3[format.arg]{Class template \tcode{basic_format_arg}}
%
%\indexlibraryglobal{basic_format_arg}%
%\begin{codeblock}
%namespace std {
%template<class Context>
%class basic_format_arg {
%    public:
%    class handle;
%
%    private:
%    using char_type = typename Context::char_type;  // \expos
%
%    variant<monostate, bool, char_type,
%        int, unsigned int, long long int, unsigned long long int,
%        float, double, long double,
%        const char_type*, basic_string_view<char_type>,
%        @\added{const char8\_t*, basic\_string\_view<char8\_t>,}@
%        @\added{const char16\_t*, basic\_string\_view<char16\_t>,}@
%        @\added{const char32\_t*, basic\_string\_view<char32\_t>,}@
%        const void*, handle
%    > value; // \expos
%
%    template<class T> explicit basic_format_arg(T& v) noexcept; // \expos
%
%    public:
%    basic_format_arg() noexcept;
%
%    explicit operator bool() const noexcept;
%
%    template<class Visitor>
%    decltype(auto) visit(this basic_format_arg arg, Visitor&& vis);
%    template<class R, class Visitor>
%    R visit(this basic_format_arg arg, Visitor&& vis);
%};
%}
%\end{codeblock}
%
%\pnum
%An instance of \tcode{basic_format_arg} provides access to
%a formatting argument for user-defined formatters.
%
%\pnum
%The behavior of a program that adds specializations of
%\tcode{basic_format_arg} is undefined.
%
%\indexlibrary{\idxcode{basic_format_arg}!constructor|(}%
%\begin{itemdecl}
%    basic_format_arg() noexcept;
%\end{itemdecl}
%
%\begin{itemdescr}
%    \pnum
%    \ensures
%    \tcode{!(*this)}.
%\end{itemdescr}
%
%\begin{itemdecl}
%    template<class T> explicit basic_format_arg(T& v) noexcept;
%\end{itemdecl}
%
%\begin{itemdescr}
%\pnum
%\constraints
%\tcode{T} satisfies \tcode{\exposconcept{formattable-with}<Context>}.
%
%\pnum
%\expects
%If \tcode{decay_t<T>} is \tcode{char_type*} \changed{or}{,} \tcode{const char_type*},
%\added{or a pointer to a possibly const-qualified Unicode character type,} \tcode{static_cast<const char_type*>(v)} points to a NTCTS \iref{defns.ntcts}.
%
%\pnum
%\effects
%Let \tcode{TD} be \tcode{remove_const_t<T>}.
%\begin{itemize}
%    \item
%    If \tcode{TD} is \tcode{bool} or \tcode{char_type},
%    initializes \tcode{value} with \tcode{v};
%    \item
%    otherwise, if \tcode{TD} is \tcode{char} and \tcode{char_type} is
%    \keyword{wchar_t}, initializes \tcode{value} with
%    \tcode{static_cast<wchar_t>(static_cast<unsigned char>(v))};
%    \begin{addedblock}
%    \item
%    otherwise, if \tcode{TD} is \added{a Unicode character type},
%    initializes \tcode{value} with \tcode{v};
%    \end{addedblock}
%    \item
%    otherwise, if \tcode{TD} is a signed integer type \iref{basic.fundamental}
%    and \tcode{sizeof(TD) <= sizeof(int)},
%    initializes \tcode{value} with \tcode{static_cast<int>(v)};
%    \item
%    otherwise, if \tcode{TD} is an unsigned integer type and
%    \tcode{sizeof(TD) <= sizeof(unsigned int)}, initializes
%    \tcode{value} with \tcode{static_cast<unsigned int>(v)};
%    \item
%    otherwise, if \tcode{TD} is a signed integer type and
%    \tcode{sizeof(TD) <= sizeof(long long int)}, initializes
%    \tcode{value} with \tcode{static_cast<long long int>(v)};
%    \item
%    otherwise, if \tcode{TD} is an unsigned integer type and
%    \tcode{sizeof(TD) <= sizeof(unsigned long long int)}, initializes
%    \tcode{value} with
%    \tcode{static_cast<unsigned long long int>(v)};
%    \item
%    otherwise, if \tcode{TD} is a standard floating-point type,
%    initializes \tcode{value} with \tcode{v};
%    \item
%    otherwise, if \tcode{TD} is
%    a specialization of \tcode{basic_string_view} or \tcode{basic_string} and
%    \tcode{TD::value_type} is \tcode{char_type} \added{or a Unicode character type},
%    initializes \tcode{value} with
%    \tcode{basic_string_view<\changed{char_type}{TD}>(v.data(), v.size())};
%    \item
%    otherwise, if \tcode{decay_t<TD>} is
%    \tcode{char_type*}\changed{ or}{,} \tcode{const char_type*}, \added{or a pointer to a possibly const-qualified Unicode character type}
%    initializes \tcode{value} with \tcode{static_cast<const \changed{char_type}{decay_t<TD>}*>(v)};
%    \item
%    otherwise, if \tcode{is_void_v<remove_pointer_t<TD>>} is \tcode{true} or
%    \tcode{is_null_pointer_v<TD>} is \tcode{true},
%    initializes \tcode{value} with \tcode{static_cast<const void*>(v)};
%    \item
%    otherwise, initializes \tcode{value} with \tcode{handle(v)}.
%\end{itemize}
%\begin{note}
%    Constructing \tcode{basic_format_arg} from a pointer to a member is ill-formed
%    unless the user provides an enabled specialization of \tcode{formatter}
%    for that pointer to member type.
%\end{note}
%\end{itemdescr}


\section{Feature test macros}

\ednote{ In <format>, bump \tcode{__cpp_lib_format_uchar} to the date of adoption}.


\section{Acknowledgments}

I'd like to thank Victor Zverovich for the incredible work on \tcode{std::format},
and SG-16 for motivating me to write this paper.
Thanks to Mark de Wever and Jonathan Wakely to help me understand the use cases and extensibility limitations
of \tcode{basic_format_arg}

\bibliographystyle{plain}
\bibliography{wg21}


\renewcommand{\section}[2]{}%

\begin{thebibliography}{9}

\bibitem[N4958]{N4958}
Thomas Köppe
\emph{Working Draft, Standard for Programming Language C++}\newline
\url{https://wg21.link/N4958}


\end{thebibliography}
\end{document}
